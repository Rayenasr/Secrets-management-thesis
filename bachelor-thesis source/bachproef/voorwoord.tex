%%=============================================================================
%% Voorwoord
%%=============================================================================

\chapter*{\IfLanguageName{dutch}{Woord vooraf}{Preface}}
\label{ch:voorwoord}

%% TODO:
%% Het voorwoord is het enige deel van de bachelorproef waar je vanuit je
%% eigen standpunt (``ik-vorm'') mag schrijven. Je kan hier bv. motiveren
%% waarom jij het onderwerp wil bespreken.
%% Vergeet ook niet te bedanken wie je geholpen/gesteund/... heeft
%Het volgende onderzoek werd geschreven in het kader van de opleiding Toegepaste Informatica met afstudeerrichting Systeem en Netwerkbeheer. 


%\lipsum[1-2]
Deze thesis ``Secrets management: centraal beheer van gevoelige data`` werd geschreven in samenwerking met het development team van Wolters Kluwer Financial Services. Deze thesis dient als informatieve bron \& als handleiding om de stappen te reproduceren voor te behalen wat met deze proef verwezenlijkt werd. Ik heb dit onderwerp gekozen omdat ik wist dat de resultaten potentieel zouden helpen bij het beheer van gevoelige data doorheen het IT-ecosysteem van Wolters Kluwer.

Deze proef vormt dan ook een hoogtepunt tijdens mijn laatste stappen om mijn studies van de opleiding ``Toegepaste Informatica met afstudeerrichting Systeem en Netwerkbeheer``, te voltooien. Het gekozen onderwerp was zeer interessant om me hierin te verdiepen en kwam met veel uitdagingen die getackeld moesten worden. Dit alles zou ik nooit volbracht hebben zonder de hulp van een aantal personen en hiervoor neem ik graag de tijd om deze mensen te bedanken.

Allereerst wil ik mijn promotor, Antonia Pierreux, bedanken voor alle hulp die ze mij heeft aangeboden door middel van meerdere feedback momenten en mijn teksten meerdere keren door te nemen waarmee ik aanpassingen kon verrichten waar nodig. Ze heeft mij ook sterk gesteund om de juiste beslissingen op tijd te nemen.

Ten tweede wil ik ook graag mijn co-promotor, Jan Delamper \& Herman van der Merwe bedanken voor alle hulp tijdens mijn stage bij Wolters Kluwer \& de mogelijkheid om de proof of concept uit te werken via middelen van het bedrijf.

Als laatste wou ik zeker ook mijn vrienden en medestudenten, Owen van Damme, Emiel van Belle, Olivier Troch en Denys Slyvka  bedanken voor de morele steun die mij heeft blijven motiveren, en de bruikbare feedback die ik heb verwerkt in deze thesis.